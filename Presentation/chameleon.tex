\documentclass{beamer}
\usetheme[pageofpages=of,% String used between the current page and the
                         % total page count.
          bullet=circle,% Use circles instead of squares for bullets.
          titleline=true,% Show a line below the frame title.
          alternativetitlepage=true,% Use the fancy title page.
          titlepagelogo=Immagini/logo-unitn,% Logo for the first page.
          ]{Torino}

\author{Stefano Leonardi\newline Enrico Saccon}
\title{A path planning for Rob.}
%\institute{}
\date{University of Trento}


\definecolor{Ashgrey}{rgb}{0.7, 0.75, 0.71}
\definecolor{Atompurple}{rgb}{0.6, 0.4, 0.8}
\makeatletter
\newcommand{\unsize}{\@setfontsize{\unsize}{11pt}{11pt}}
\makeatother
\makeatletter
\newcommand{\disize}{\@setfontsize{\disize}{10pt}{10pt}}
\makeatother
\makeatletter
\newcommand{\nosize}{\@setfontsize{\nosize}{9pt}{9pt}}
\makeatother
\makeatletter
\newcommand{\otsize}{\@setfontsize{\otsize}{8pt}{8pt}}
\makeatother
\makeatletter
\newcommand{\stsize}{\@setfontsize{\stsize}{7pt}{7pt}}
\makeatother
\makeatletter
\newcommand{\sesize}{\@setfontsize{\sesize}{6pt}{6pt}}
\makeatother
\makeatletter
\newcommand{\cisize}{\@setfontsize{\cisize}{5pt}{5pt}}
\makeatother
\makeatletter
\newcommand{\qusize}{\@setfontsize{\qusize}{4pt}{4pt}}
\makeatother

\makeatletter
\newcommand{\srcnorm}{\@setfontsize{\srcsize}{11pt}{11pt}}
\makeatother
\makeatletter
\newcommand{\srcsize}{\@setfontsize{\srcsize}{5pt}{5pt}}
\makeatother
\makeatletter
\newcommand{\srcmint}{\@setfontsize{\srcmint}{7pt}{7pt}}
\makeatother

\usepackage{xcolor}

\usepackage{caption}
\usepackage{listings}
\usepackage[cache=false]{minted} 
\setminted{tabsize=4, breaklines, breakanywhere, mathescape}

\usepackage{tikz}
\usetikzlibrary{graphdrawing}
\usetikzlibrary{graphs}
\usetikzlibrary{arrows,automata}
\usegdlibrary{trees}

\usepackage{pgfplotstable}
\usepackage{pgfplots}
\usepackage{pgfmath}

\begin{document}
\lstset{
	  breakatwhitespace=false,         
	  breaklines=true,     
	  basicstyle=\srcsize\ttfamily,            
	  commentstyle=\color{Ashgey}, %Indica il colore dei commenti
	  keywordstyle=\color{Atompurple}, %Indica il colore delle parole chiave
	  language=C++, %Indica il linguaggio predefinito da usare
	  rulecolor=\color{black}, %Indica il colore dei numeri di righe
	  tabsize=2,
	  escapeinside={\%*}{*)},
	  morekeywords={compute, from, throw, std, ostringstream, __LINE__, __FILE__, define}, %Altre parole da inserire tra le keywords. Ad esempio possiamo aggiungere do, gotttto, ecc ecc 
}


	%TODO no numeri a lato del codice
\begin{frame}[t,plain]
\titlepage
\end{frame}

\begin{frame}{Steps}
	\begin{itemize}
		\item Calibration
		\item Unwrapping
		\item Detection
	\end{itemize}
\end{frame}


\begin{frame}{Calibration}
	Given a set of pictures taken with a camera, the object of this step is to find the camera matrix $A$ and the distortion coefficients $\dot{d}$.\newline
	\vfill
	\[
		A=\begin{bmatrix}
			f_x&0&c_x\\
			0&f_y&c_y\\
			0&0&1\\
		\end{bmatrix}\qquad
		\begin{matrix}
			x_{distorted}=x\left(1+k_1r^2+k_2r^4+k_3r^6\right)\\
			y_{distorted}=y\left(1+k_1r^2+k_2r^4+k_3r^6\right)\\
			x_{distorted}=x+\left[2p_1xy+p_2\left(r^2+2x^2\right)\right]\\
			y_{distorted}=y+\left[p_1\left(r^2+2y^2\right)+2p_2xy\right]\\
			\dot{d}=(k_1, k_2, p_1, p_2, k_3)
		\end{matrix}
	\]
	\vfill
\end{frame}


\begin{frame}{Calibration}
	\begin{figure}[H]
		\begin{minipage}{0.48\linewidth}
			\includegraphics[width=\linewidth]{Immagini/Chessboard1}
		\end{minipage}
		\vspace{0.04\linewidth}
		\begin{minipage}{0.48\linewidth}
			\includegraphics[width=\linewidth]{Immagini/Chessboard2}
		\end{minipage}
	\end{figure}
\end{frame}


\begin{frame}{Calibration}
	\begin{figure}[H]
		\begin{minipage}{0.48\linewidth}
			\includegraphics[width=\linewidth]{Immagini/Chessboard1Calibrated}
		\end{minipage}
		\vspace{0.04\linewidth}
		\begin{minipage}{0.48\linewidth}
			\includegraphics[width=\linewidth]{Immagini/Chessboard2Calibrated}
		\end{minipage}
	\end{figure}
	\stsize
	\[A=\begin{bmatrix}
		8.4247565095622963\times10^2 & 0 & 6.3709750745251142\times10^2\\
		0 & 8.4247565095622963\times10^2 & 4.9404840840221556\times10^2\\
		0 & 0 & 1
	\end{bmatrix}\]\[
	\dot{d}=(-2.5214446354851400\times10^{-1}, 7.2467634259951161\times10^{-2},\]\[-3.7212601356153754\times10^{-3}, 4.3313659139950872\times10^{-4}, 0)
	\]
\end{frame}

\begin{frame}{Unwrapping}
	
\end{frame}













\end{document}

